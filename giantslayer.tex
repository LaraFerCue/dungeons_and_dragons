\archetype{Giant Slayer}
Giants are evil creatures that hunt and devour humanoids. There are a few of them that are good, but it is safe to assume that a giant will not stop and talk. A giant slayer has learnt to get advantage of this large creatures and use their size as an advantage.

\begin{archetypespells}
	3rd & \textit{Grease, Longstrider} \\
	7th & \textit{Enlarge/Reduce, Mirror Image} \\
	11th & \textit{Haste, Slow}\\
	15th & \textit{Black Tentacles, Fire Shield}\\
\end{archetypespells}

\subsubsection*{Colossal hunter}
Starting at 3rd level, you can use a weapon attack on a Large or larger creature to make them fall. On a hit, you cause your regular damage and the creature must succeed on a Strength saving throw or fall prone. The DC on this saving throw is 8 + your attack modifier with this weapon.
\subsubsection*{Defensive Tactics}
Starting a 5th level, you can choose one of the following features:

\textbf{Quick reflexes.} You have learnt to avoid boulders and other ranged attacks from this creatures. Your AC is incremented by 2 when a Large or larger creature attempts to hit you with a range weapon attack.

\textbf{Deathly speed.} The best way to avoid a giant to smash you, is to be quick. You can use your bonus action to \textit{Disengage}.
\subsubsection*{Mark of the Slayer}
Starting at 7th level, you can use your bonus action to mark a Large or larger creature. You and your friends have advantage on weapon attacks against the creature. You point the weaknesses of your foe.
\subsubsection*{Colossal slayer}
At level 11th, you can cause an additional 1d10 to a Large or larger creature once per turn.
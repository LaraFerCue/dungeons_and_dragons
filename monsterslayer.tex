\documentclass[11pt,a4paper,twocolumn, sans]{article}
\usepackage{bibcheck}
\usepackage{graphicx}
\usepackage{wrapfig}
\usepackage{tabularx}
\usepackage[table]{xcolor}

\newcommand{\proficiency}[2]
{
	\textbf{#1:} #2. \\
}

\title{Monster Slayer}
\author{}
\date{}
\begin{document}
	\maketitle
	\section*{Protectors of Humanity}
	The monster slayer is a warrior specialized in slaying vampires, lycantropes, demons and other supernatural creatures. The monster slayers train with clerics and paladins	learning their ways, but instead of offering a direct aproach, they like to track their prey and finish them using hunter's strategy: plan traps, get the target where they want them to be and 	end the fight quickly using divine magic combined with quick attacks.
	\section*{Independent adventurers}
	A monster slayer is always afraid of getting innocent people to enlarge the undead's armies and they always tend to avoid rallying with non holy members. Their sacred duty to protect humanity from this evil forces, make them paranoid about	other humanoids being charmed or dominated by these creatures. \\
	Although, slayers are not to be quiet waiting the evil to come to their doors. They usually	march seeking for contracts to finish these monsters or accepting charity when needed to protect some villagers.
	\section*{Creating a monster slayer}
	As you create your slayer character, consider the nature of the training that gave you your particular capabilities. Did you train with a single mentor, wandering the lands together until you mastered the monster slayer’s ways? Did you leave your apprenticeship, or was your mentor slain— perhaps by the same kind of monster that became your favored enemy? Or perhaps you learned your skills as part of a band of slayers affiliated with a monastery, trained in mystic paths as well as sacred lore. You might be self-taught, a recluse who learned combat skills, tracking, and even a magical connection to the gods through the experience in slaying undeads and other monsters. \\ \\
	What’s the source of your particular hatred of a certain kind of enemy? Did a monster kill someone you loved or destroy your home village? Or did you see too much of the destruction these monsters cause and commit yourself to reining in their depredations? Is your adventuring career a continuation of your work in protecting the borderlands, or a significant change? What made you join up with a band of adventurers? Do you find it challenging to teach new allies the ways of the wild, or do you welcome the relief from solitude that they offer?
	\section*{Quick Build}
	You can make a monster slayer quickly by following these suggestions. First, make Dexterity your highest score, followed by Wisdom. (Some witchers who focus on two-weapon fighting make Strength higher than Dexterity.) Second, choose the soldier background.
	
	\begin{table*}
		\small
		\rowcolors{1}{white}{lightgray}
		\hrule
		\begin{tabularx}{\textwidth}{|l p{30pt} p{180pt} p{30pt} X X X X X|
			}
			Level & Prof. Bonus & Features & Spell Known & \multicolumn{5}{c|}{-- Spell Slots / Spell Level --} \\
			1st & +2 & Favored Enemy, Divine Hunter & - & - & - & - & - &- \\
			2nd & +2 & Fighting Style, Spell Casting & 2 & 2 & - & - &- &- \\
			3rd & +2 & Monster Slayer Archetype, Primeval Awareness & 3 & 3 & - & - &- &- \\
			4th & +2 & Ability Score Improvement & 3 & 3 & - & - &- &- \\
			5th & +3 & Extra Attack & 4 & 4 & 2 & - & - & - \\
			6th & +3 & Favored Enemy and Divine Hunter Improvement & 4 & 4 & 3 & - & - & - \\
			7th & +3 & Monster Slayer Archetype feature & 5 & 4 & 3  & - & - & - \\
			8th & +3 & Ability Score Improvement & 5 & 4 & 3  & - & - & - \\
			9th & +4 & Slayer of Magic Wielders & 6 & 4 & 3 & 2 & - & - \\
			10th & +4 & Divine Hunter Improvement & 6 & 4 & 3 & 2 & - & - \\
			11th & +4 & Monster Slayer Archetype feature & 7 & 4 & 3  & 3 & - & - \\
			12th & +4 & Ability Score Improvement & 7 & 4 & 3  & 3 & - & - \\
			13th & +5 & - & 8 & 4 & 3 & 3 & 1 & - \\
			14th & +5 & Favorite Enemy improvement, Vanish & 8 & 4 & 3 & 3 & 1 & - \\
			15th & +5 & Monster Slayer Archetype feature & 9 & 4 & 3 & 3 & 2 & - \\
			16th & +5 & Ability Score Improvement & 9 & 4 & 3 & 3 & 2 & - \\
			17th & +6 & - & 10 & 4 & 3 & 3 & 3 & 1 \\
			18th & +6 & Feral Senses & 10 & 4 & 3 & 3 & 3 & 1 \\
			19th & +6 & Ability Score Improvement & 11 & 4 & 3 & 3 & 3 & 2 \\
			20th & +6 & Foe Slayer & 11 & 4 & 3 & 3 & 3 & 2 \\
		\end{tabularx}
		\hrule
	\end{table*}
	\section*{Class Features}
	As a monster slayer, you gain the following class features.
	
	\subsection*{Hit Points}
	\proficiency{Hit Dice}{1d10 per monster slayer level}
	\proficiency{Hit Points at 1st level}{10 + your Constitution modifier}
	\proficiency{Hit Points at Higher Levels}{1d10 (or 6) + your Constitution modifier per monster slayer level after 1st}
	
	\subsection*{Proficiencies}
	\proficiency{Armor}{light armor, medium armor, shields}
	\proficiency{Weapons}{Simple weapons and martial weapons}
	\proficiency{Tools}{none}
	\proficiency{Saving Throws}{Dexterity, Wisdom}
	\proficiency{Skills}{Choose three from Animal Handling, Athletics, Insight, Investigation, Religion, Perception, Stealth, Survival}
	
	\subsection*{Equipment}
	You start with the following equipment, in
	addition to the equipment granted by your
	background:
	\begin{itemize}
		\item (a) Scale mail or (b) leather armor.
		\item (a) Two shortswords or (b) two simple weapons.
		\item (a) a dungeoneer's pack or (b) an explorer's pack.
		\item a longbow and a quiver of 20 arrows.
		\item a holy symbol.
	\end{itemize}
	
\end{document}